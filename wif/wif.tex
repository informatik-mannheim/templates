% !TeX program = lualatex

% Dokumenteneinstellungen und Infos importieren
% -------------------------------------------------------
% Informationen und Einstellungen für das Paper in dem Modul WIF
%

% Sprache für das Dokument festlegen
\newcommand{\hsmasprache}{de} %de für Deutsch oder en für Englisch

%Angabe des aktuellen Semester
\newcommand{\currentsemester}{WS} %  SoSe


\newcommand{\hsmatitel}{Das ist ein wirklich langer Titel für das wissenschaftliche Paper in dem Modul wissenschaftliches Arbeiten für Fortgeschrittene im Masterstudiengang Informatik Software Engineering und Data Science} %Lange Version des Titels
\newcommand{\hsmatitelshort}{ Kurze Version Titel} %Kurzversion des Titels
\newcommand{\hsmaautorvname}{Max}        % Vorname(n)
\newcommand{\hsmaautornname}{Mustermann} % Nachname(n)


% Abstrakt
\newcommand{\hsmaabstract}{Dies ist die Zusammenfassung dieses Dokuments. Daher dieses Dokument nur zu Testzwecken dient, fällt die Zusammenfassung entsprechend kurz aus. Dies ist die Zusammenfassung dieses Dokuments. Daher dieses Dokument nur zu Testzwecken dient, fällt die Zusammenfassung entsprechend kurz aus.Dies ist die Zusammenfassung dieses Dokuments. Daher dieses Dokument nur zu Testzwecken dient, fällt die Zusammenfassung entsprechend kurz aus.	Dies ist die Zusammenfassung dieses Dokuments. Daher dieses Dokument nur zu Testzwecken dient, fällt die Zusammenfassung entsprechend kurz aus.}


\newcommand{\hsmakeywords}{DDoS, Networking, Forensics}
\documentclass[\hsmasprache]{HMA}

\addbibresource{literatur.bib}


\begin{document}
	
			
	\section{Ein erster Abschnitt}
	
	%%% Nachfolgenden Befehl nicht löschen oder ändern!
	\thispagestyle{headings}	
	
	
	\blindtext[5]
	
	
	
	%%% Nachfolgende 2 Befehle nicht löschen oder ändern!
	\markright{\@\hsmaautor: \@\shorttitle}
	\thispagestyle{headings}
	
	
	
	
	
	
	\begin{figure}[th]
		\centering
		\includegraphics[width=0.7\linewidth]{bilder/nasa_rover}
		\caption{}
		\label{fig:nasarover}
	\end{figure}
	
	\cite{Kornmeier2011}
	\blindtext[4]
	\subsection{Unterabschnitt}
	
	\blindtext[5]
	
	\subsubsection{Unter-Unterabschnitt}
	
	\blindtext[5]
	
	\paragraph{Eine weitere Überschrift}
	
	\begin{figure*}
		\centering
		\includegraphics[width=0.7\linewidth]{bilder/nature}
		\caption{}
		\label{fig:nature}
	\end{figure*}
	
	
	\subparagraph{Noch eine kleinere Überschrift}
	
	\subsection{Weiterer Unterabschnitt}
	
	\begin{align*} 
		2x - 5y &=  8 \\ 
		3x + 9y &=  -12
	\end{align*}
	
	\subsubsection{Unter-Unterabschnitt}
	
	
	\begin{enumerate}
		\item Eintrag 1
		\item Eintrag 2
		\item Eintrag 3
	\end{enumerate}
	
	\section{Tabellen}
	
	Tabellen werden normalerweise ohne vertikale Striche gesetzt, sondern die Spalten werden durch einen entsprechenden Abstand voneinander getrennt.\footnote{Siehe \cite[S. 89]{Willberg2021}.} Zum Einsatz kommen ausschließlich horizontale Linien (siehe \autoref{Kap2:Kopplungsformen}).
	
	\begin{table}[ht]
		\caption{Ebenen der Kopplung und Beispiele für enge und lose Kopplung}
		\label{Kap2:Kopplungsformen}
		\renewcommand{\arraystretch}{1.2}
		\centering
		\resizebox{\linewidth}{!}{  
			\begin{tabular}{l l l}
				\toprule
				\textbf{Form der Kopplung} & \textbf{enge Kopplung} & \textbf{lose Kopplung}\\
				\midrule
				Physikalische Verbindung	&	Punkt-zu-Punkt	& 	über Vermittler\\
				Kommunikationsstil	&	synchron		&	asynchron\\
				Datenmodell	&	komplexe gemeinsame Typen	&	nur einfache gemeinsame Typen\\
				Bindung	&	statisch		&	dynamisch\\
				\bottomrule
			\end{tabular}
		}
	\end{table}
	
	
	\begin{table*}[ht]
		\caption{Ebenen der Kopplung und Beispiele für enge und lose Kopplung}
		\label{Kap2:KopplungsformenA}
		\renewcommand{\arraystretch}{1.2}
		\centering
			\begin{tabular}{l l l}
				\toprule
				\textbf{Form der Kopplung} & \textbf{enge Kopplung} & \textbf{lose Kopplung}\\
				\midrule
				Physikalische Verbindung	&	Punkt-zu-Punkt	& 	über Vermittler\\
				Kommunikationsstil	&	synchron		&	asynchron\\
				Datenmodell	&	komplexe gemeinsame Typen	&	nur einfache gemeinsame Typen\\
				Bindung	&	statisch		&	dynamisch\\
				\bottomrule
			\end{tabular}
	\end{table*}
	
	\appendix
	
	\section{Der erste Anhang}
	
	\subsection{Mit einem Unteranschnitt}
	
	


\end{document}